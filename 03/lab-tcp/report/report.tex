\documentclass[a4paper,12pt]{article}

\input{header.tex}

\title{Отчёт по лабораторной работе \\ <<Механизмы протокола TCP>>}
\author{Binh D. Nguyen}

\begin{document}

\maketitle

\tableofcontents

% Текст отчёта должен быть читаемым!!! Написанное здесь является рыбой.

\section{Установка и разрыв соединения}

Дампим интерфейсы с2 и с3 при попытке соединения с1  и с6 c cервером

\begin{Verbatim}
c1:~# telnet 10.50.0.2 9
Trying 10.50.0.2...
Connected to 10.50.0.2.
Escape character is '^]'.
123
1234  

c6:~# telnet 10.50.0.2 9

c2:~# tcpdump -n -i eth0 tcp
tcpdump: verbose output suppressed, use -v or -vv for full protocol decode
listening on eth0, link-type EN10MB (Ethernet), capture size 96 bytes
20:18:34.163563 IP 10.10.0.2.40128 > 10.50.0.2.9: P 2641696858:2641696863(5) ack 2652137935 win 2920 <nop,nop,timestamp 4294949596 4294946630>
20:18:34.164177 IP 10.50.0.2.9 > 10.10.0.2.40128: . ack 5 win 2896 <nop,nop,timestamp 4294949592 4294949596>
20:24:10.734964 IP 10.10.0.2.40128 > 10.50.0.2.9: P 5:15(10) ack 1 win 2920 <nop,nop,timestamp 15957 4294949592>
20:24:10.736084 IP 10.50.0.2.9 > 10.10.0.2.40128: . ack 15 win 2896 <nop,nop,timestamp 15953 15957>

c3:~# tcpdump -n -i eth0 tcp
tcpdump: verbose output suppressed, use -v or -vv for full protocol decode
listening on eth0, link-type EN10MB (Ethernet), capture size 96 bytes
20:18:34.150238 IP 10.10.0.2.40128 > 10.50.0.2.9: P 2641696858:2641696863(5) ack 2652137935 win 2920 <nop,nop,timestamp 4294949596 4294946630>
20:18:34.150600 IP 10.50.0.2.9 > 10.10.0.2.40128: . ack 5 win 2896 <nop,nop,timestamp 4294949592 4294949596>
20:24:10.721711 IP 10.10.0.2.40128 > 10.50.0.2.9: P 5:15(10) ack 1 win 2920 <nop,nop,timestamp 15957 4294949592>
20:24:10.722453 IP 10.50.0.2.9 > 10.10.0.2.40128: . ack 15 win 2896 <nop,nop,timestamp 15953 15957>

\end{Verbatim}

\section{Окно получателя}

Где что дампим.  Дампить без -t обязательно!

\begin{Verbatim}
окно получателя до нуля
\end{Verbatim}

\section{Окно отправителя}

Где что дампим.

\begin{Verbatim}
окно отправителя растёт и растёт
\end{Verbatim}

\section{Нейгл и Мишналь}

Где что дампим.  Дампить без -t обязательно!
Мы должны увидеть, что мы посылаем неполный сегмент, как только подтверждены все неполные сегменты (даже если полные не подтверждены).

\begin{Verbatim}
увидеть хотя бы нейгла
\end{Verbatim}

\section{Аггрессивная буферизация}

Где что дампим.  Дампить без -t обязательно!

\begin{Verbatim}
cork
\end{Verbatim}

\section{Отправка без задержки}

Где что дампим.  Дампить без -t обязательно!

\begin{Verbatim}
увидеть, что nodelay не помогает, когда окно отправителя полное
\end{Verbatim}

\section{Быстрый повтор}

Где что дампим.  Дампить без -t обязательно!

\begin{Verbatim}
увидеть быстрый повтор
\end{Verbatim}

\section{Обычный повтор}

Где что дампим. Дампить без -t обязательно!

\begin{Verbatim}
увидеть не-быстрый повтор
\end{Verbatim}

\section{Неудачная попытка соединени с портом}

Тут опт с попыткой соединения с портом, который никто не слушает.

\section{Опыт с PMTU}

Для опыта нужно на c3 и c4 отключить tcpclump.
Надеюсь тут поможет команда iptables -F.
Для сброса кеша MSS на с1 его можно тупо перегрузить.

После этих подготовительных действий мы наверное увидим pmtu в действии при попытке соединится с c1 на c6.

\begin{Verbatim}
Дампить (tcp or icmp) на c2 eth0.
\end{Verbatim}

\section{Соединение с неверным портом}

Что будет, если клиент пытается соединиться с портом, который не слушвет сервер.

\begin{Verbatim}
Дампить
\end{Verbatim}

\end{document}
